\documentclass[titlepage,a4paper]{article}

\usepackage{a4wide}
\usepackage[colorlinks=true,linkcolor=black,urlcolor=blue,bookmarksopen=true]{hyperref}
\usepackage{bookmark}
\usepackage{fancyhdr}
\usepackage[spanish]{babel}
\usepackage[utf8]{inputenc}
\usepackage[T1]{fontenc}
\usepackage{graphicx}
\usepackage{adjustbox}
\usepackage{float}

\pagestyle{fancy} % Encabezado y pie de página
\fancyhf{}
\fancyhead[L]{TP1}
\fancyhead[R]{FIUBA}
\renewcommand{\headrulewidth}{0.4pt}
\fancyfoot[C]{\thepage}
\renewcommand{\footrulewidth}{0.4pt}

\begin{document}
\begin{titlepage} % Carátula
    \centering
    \vfill
    \Huge \textbf{Trabajo Práctico 1} \\
    
    \vskip1cm
    \Huge [75.06] Organización de Datos\\
    \vskip2cm
    \begin{table}[htbp]
	\begin{center}
	\begin{adjustbox}{max width=\textwidth}
	\begin{tabular}{|l|l|l|}
	\hline
    \multicolumn{3}{|c|}{Grupo 9} \\ \hline
	Nombre & Padrón & Mail \\ \hline 
    Leonardo Bellaera & 100973 & leobellaera@gmail.com \\ \hline 
    Nicolás Bugliot & pad  & mail \\ \hline
    Filyan Karagoz &  pad & mail \\ \hline
    Alejandro Kler &  pad & mail \\ \hline
    \end{tabular}
    \end{adjustbox}
    \vskip 2cm
    \includegraphics[width=8.7cm, height=8.7cm]{UBA.png}
	\label{tabla:sencilla}
	\end{center}
	\end{table}
	\vskip1cm
	\Large \textit{Primer cuatrimestre de 2019}

    \vfill
\end{titlepage}
\tableofcontents % Índice general
\newpage

\section{Introducción}\label{sec:intro}

(acá va todo el chamuyo de la introducción...)

\newpage

\section{Análisis exploratorio y visualización de los datos}\label{sec:intro}

En esta sección se desarrolla el análisis exploratorio realizado sobre los datos (..agregar cosas)

\newpage  %cada análisis lo ponemos en una hoja nueva

\subsection{Análisis de x cosa}

Acá pones todo el análisis q realizaste sobre la variable x de los datasets incluyendo gráficos y demás.

\newpage 

\subsection{Análisis de y cosa}

Acá pones todo el análisis q realizaste sobre la variable y de los datasets incluyendo gráficos y demás.

\newpage  %la conclusión la apartamos del analisis

\section{Conclusión}\label{sec:intro}

Acá va todo el chamuyo de la conclusión (...)


\end{document}